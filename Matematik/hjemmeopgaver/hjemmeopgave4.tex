% Options for packages loaded elsewhere
\PassOptionsToPackage{unicode}{hyperref}
\PassOptionsToPackage{hyphens}{url}
\PassOptionsToPackage{dvipsnames,svgnames,x11names}{xcolor}
%
\documentclass[
  a4paper,
]{article}

\usepackage{amsmath,amssymb}
\usepackage{iftex}
\ifPDFTeX
  \usepackage[T1]{fontenc}
  \usepackage[utf8]{inputenc}
  \usepackage{textcomp} % provide euro and other symbols
\else % if luatex or xetex
  \usepackage{unicode-math}
  \defaultfontfeatures{Scale=MatchLowercase}
  \defaultfontfeatures[\rmfamily]{Ligatures=TeX,Scale=1}
\fi
\usepackage{lmodern}
\ifPDFTeX\else  
    % xetex/luatex font selection
\fi
% Use upquote if available, for straight quotes in verbatim environments
\IfFileExists{upquote.sty}{\usepackage{upquote}}{}
\IfFileExists{microtype.sty}{% use microtype if available
  \usepackage[]{microtype}
  \UseMicrotypeSet[protrusion]{basicmath} % disable protrusion for tt fonts
}{}
\makeatletter
\@ifundefined{KOMAClassName}{% if non-KOMA class
  \IfFileExists{parskip.sty}{%
    \usepackage{parskip}
  }{% else
    \setlength{\parindent}{0pt}
    \setlength{\parskip}{6pt plus 2pt minus 1pt}}
}{% if KOMA class
  \KOMAoptions{parskip=half}}
\makeatother
\usepackage{xcolor}
\usepackage[top=2.5cm,bottom=2.5cm,left=2cm,right=2cm]{geometry}
\setlength{\emergencystretch}{3em} % prevent overfull lines
\setcounter{secnumdepth}{-\maxdimen} % remove section numbering
% Make \paragraph and \subparagraph free-standing
\ifx\paragraph\undefined\else
  \let\oldparagraph\paragraph
  \renewcommand{\paragraph}[1]{\oldparagraph{#1}\mbox{}}
\fi
\ifx\subparagraph\undefined\else
  \let\oldsubparagraph\subparagraph
  \renewcommand{\subparagraph}[1]{\oldsubparagraph{#1}\mbox{}}
\fi

\usepackage{color}
\usepackage{fancyvrb}
\newcommand{\VerbBar}{|}
\newcommand{\VERB}{\Verb[commandchars=\\\{\}]}
\DefineVerbatimEnvironment{Highlighting}{Verbatim}{commandchars=\\\{\}}
% Add ',fontsize=\small' for more characters per line
\usepackage{framed}
\definecolor{shadecolor}{RGB}{241,243,245}
\newenvironment{Shaded}{\begin{snugshade}}{\end{snugshade}}
\newcommand{\AlertTok}[1]{\textcolor[rgb]{0.68,0.00,0.00}{#1}}
\newcommand{\AnnotationTok}[1]{\textcolor[rgb]{0.37,0.37,0.37}{#1}}
\newcommand{\AttributeTok}[1]{\textcolor[rgb]{0.40,0.45,0.13}{#1}}
\newcommand{\BaseNTok}[1]{\textcolor[rgb]{0.68,0.00,0.00}{#1}}
\newcommand{\BuiltInTok}[1]{\textcolor[rgb]{0.00,0.23,0.31}{#1}}
\newcommand{\CharTok}[1]{\textcolor[rgb]{0.13,0.47,0.30}{#1}}
\newcommand{\CommentTok}[1]{\textcolor[rgb]{0.37,0.37,0.37}{#1}}
\newcommand{\CommentVarTok}[1]{\textcolor[rgb]{0.37,0.37,0.37}{\textit{#1}}}
\newcommand{\ConstantTok}[1]{\textcolor[rgb]{0.56,0.35,0.01}{#1}}
\newcommand{\ControlFlowTok}[1]{\textcolor[rgb]{0.00,0.23,0.31}{#1}}
\newcommand{\DataTypeTok}[1]{\textcolor[rgb]{0.68,0.00,0.00}{#1}}
\newcommand{\DecValTok}[1]{\textcolor[rgb]{0.68,0.00,0.00}{#1}}
\newcommand{\DocumentationTok}[1]{\textcolor[rgb]{0.37,0.37,0.37}{\textit{#1}}}
\newcommand{\ErrorTok}[1]{\textcolor[rgb]{0.68,0.00,0.00}{#1}}
\newcommand{\ExtensionTok}[1]{\textcolor[rgb]{0.00,0.23,0.31}{#1}}
\newcommand{\FloatTok}[1]{\textcolor[rgb]{0.68,0.00,0.00}{#1}}
\newcommand{\FunctionTok}[1]{\textcolor[rgb]{0.28,0.35,0.67}{#1}}
\newcommand{\ImportTok}[1]{\textcolor[rgb]{0.00,0.46,0.62}{#1}}
\newcommand{\InformationTok}[1]{\textcolor[rgb]{0.37,0.37,0.37}{#1}}
\newcommand{\KeywordTok}[1]{\textcolor[rgb]{0.00,0.23,0.31}{#1}}
\newcommand{\NormalTok}[1]{\textcolor[rgb]{0.00,0.23,0.31}{#1}}
\newcommand{\OperatorTok}[1]{\textcolor[rgb]{0.37,0.37,0.37}{#1}}
\newcommand{\OtherTok}[1]{\textcolor[rgb]{0.00,0.23,0.31}{#1}}
\newcommand{\PreprocessorTok}[1]{\textcolor[rgb]{0.68,0.00,0.00}{#1}}
\newcommand{\RegionMarkerTok}[1]{\textcolor[rgb]{0.00,0.23,0.31}{#1}}
\newcommand{\SpecialCharTok}[1]{\textcolor[rgb]{0.37,0.37,0.37}{#1}}
\newcommand{\SpecialStringTok}[1]{\textcolor[rgb]{0.13,0.47,0.30}{#1}}
\newcommand{\StringTok}[1]{\textcolor[rgb]{0.13,0.47,0.30}{#1}}
\newcommand{\VariableTok}[1]{\textcolor[rgb]{0.07,0.07,0.07}{#1}}
\newcommand{\VerbatimStringTok}[1]{\textcolor[rgb]{0.13,0.47,0.30}{#1}}
\newcommand{\WarningTok}[1]{\textcolor[rgb]{0.37,0.37,0.37}{\textit{#1}}}

\providecommand{\tightlist}{%
  \setlength{\itemsep}{0pt}\setlength{\parskip}{0pt}}\usepackage{longtable,booktabs,array}
\usepackage{calc} % for calculating minipage widths
% Correct order of tables after \paragraph or \subparagraph
\usepackage{etoolbox}
\makeatletter
\patchcmd\longtable{\par}{\if@noskipsec\mbox{}\fi\par}{}{}
\makeatother
% Allow footnotes in longtable head/foot
\IfFileExists{footnotehyper.sty}{\usepackage{footnotehyper}}{\usepackage{footnote}}
\makesavenoteenv{longtable}
\usepackage{graphicx}
\makeatletter
\def\maxwidth{\ifdim\Gin@nat@width>\linewidth\linewidth\else\Gin@nat@width\fi}
\def\maxheight{\ifdim\Gin@nat@height>\textheight\textheight\else\Gin@nat@height\fi}
\makeatother
% Scale images if necessary, so that they will not overflow the page
% margins by default, and it is still possible to overwrite the defaults
% using explicit options in \includegraphics[width, height, ...]{}
\setkeys{Gin}{width=\maxwidth,height=\maxheight,keepaspectratio}
% Set default figure placement to htbp
\makeatletter
\def\fps@figure{htbp}
\makeatother

\usepackage{icomma}
\usepackage{lastpage}
\usepackage{fancyhdr}
\pagestyle{fancy}
\fancyhf{} 
\renewcommand{\headrulewidth}{0pt}
\lhead{ Jeppe Bøgeskov 1. x }
\chead{ 01/12/2023 }
\rhead{ Matematik }
\cfoot{Side \thepage\ af \pageref*{LastPage}}
\makeatletter
\makeatother
\makeatletter
\makeatother
\makeatletter
\@ifpackageloaded{caption}{}{\usepackage{caption}}
\AtBeginDocument{%
\ifdefined\contentsname
  \renewcommand*\contentsname{Indholdsfortegnelse}
\else
  \newcommand\contentsname{Indholdsfortegnelse}
\fi
\ifdefined\listfigurename
  \renewcommand*\listfigurename{Figuroversigt}
\else
  \newcommand\listfigurename{Figuroversigt}
\fi
\ifdefined\listtablename
  \renewcommand*\listtablename{Tabeloversigt}
\else
  \newcommand\listtablename{Tabeloversigt}
\fi
\ifdefined\figurename
  \renewcommand*\figurename{Figur}
\else
  \newcommand\figurename{Figur}
\fi
\ifdefined\tablename
  \renewcommand*\tablename{Tabel}
\else
  \newcommand\tablename{Tabel}
\fi
}
\@ifpackageloaded{float}{}{\usepackage{float}}
\floatstyle{ruled}
\@ifundefined{c@chapter}{\newfloat{codelisting}{h}{lop}}{\newfloat{codelisting}{h}{lop}[chapter]}
\floatname{codelisting}{Liste}
\newcommand*\listoflistings{\listof{codelisting}{Listeoversigt}}
\makeatother
\makeatletter
\@ifpackageloaded{caption}{}{\usepackage{caption}}
\@ifpackageloaded{subcaption}{}{\usepackage{subcaption}}
\makeatother
\makeatletter
\@ifpackageloaded{tcolorbox}{}{\usepackage[skins,breakable]{tcolorbox}}
\makeatother
\makeatletter
\@ifundefined{shadecolor}{\definecolor{shadecolor}{rgb}{.97, .97, .97}}
\makeatother
\makeatletter
\makeatother
\makeatletter
\makeatother
\ifLuaTeX
\usepackage[bidi=basic]{babel}
\else
\usepackage[bidi=default]{babel}
\fi
\babelprovide[main,import]{danish}
% get rid of language-specific shorthands (see #6817):
\let\LanguageShortHands\languageshorthands
\def\languageshorthands#1{}
\ifLuaTeX
  \usepackage{selnolig}  % disable illegal ligatures
\fi
\IfFileExists{bookmark.sty}{\usepackage{bookmark}}{\usepackage{hyperref}}
\IfFileExists{xurl.sty}{\usepackage{xurl}}{} % add URL line breaks if available
\urlstyle{same} % disable monospaced font for URLs
\hypersetup{
  pdftitle={Opgave 1},
  pdflang={da},
  colorlinks=true,
  linkcolor={blue},
  filecolor={Maroon},
  citecolor={Blue},
  urlcolor={Blue},
  pdfcreator={LaTeX via pandoc}}

\title{Opgave 1}
\author{}
\date{}

\begin{document}
\maketitle
\thispagestyle{fancy}

\ifdefined\Shaded\renewenvironment{Shaded}{\begin{tcolorbox}[interior hidden, sharp corners, boxrule=0pt, borderline west={3pt}{0pt}{shadecolor}, enhanced, breakable, frame hidden]}{\end{tcolorbox}}\fi

\begin{Shaded}
\begin{Highlighting}[]
\ImportTok{from}\NormalTok{ gym\_cas }\ImportTok{import} \OperatorTok{*}
\ImportTok{import}\NormalTok{ math}
\end{Highlighting}
\end{Shaded}

Løs følgende ligninger ved ``håndregning''

\hypertarget{opgave-2}{%
\subsection{Opgave 2}\label{opgave-2}}

\hypertarget{dragejuxe6geren-lilina-er-20-uxe5r-yngre-end-ridderen-krox.-for-3-uxe5r-siden-var-lilina-halvt-suxe5-gammel-som-krox.}{%
\subparagraph{Dragejægeren Lilina er 20 år yngre end ridderen Krox. For
3 år siden var Lilina halvt så gammel som
Krox.}\label{dragejuxe6geren-lilina-er-20-uxe5r-yngre-end-ridderen-krox.-for-3-uxe5r-siden-var-lilina-halvt-suxe5-gammel-som-krox.}}

• Hvor gammel er dragejægeren Lilina?

\(L = K - 20\) \(L - 3 = \dfrac{1}{2} (K - 3)\)

Hvis hun skal være fra 3 år siden halvt så gammel som Krox og 20 år
yngre så må hun være 23 år

\hypertarget{opgave-3}{%
\subsection{Opgave 3}\label{opgave-3}}

Dragejægeren Lilina skal klatre op ad en lodret borgmur. Hun har kastet
sit 25 meter lange klatrereb op og fastgjort det i toppen (vha. sit
kasteanker). Når hun spænder klatrerebet helt ud ved at gå væk fra muren
og derefter holder rebet ned til jorden danner det en vinkel på 75∘ med
vandret.

• Hvor høj er borgmuren?

For at finde højden af borgmuren benytter vi os af Sinus.

Formel for sinus:
\(\text{Sin(V)=}\dfrac{modstående katete}{hypotenusen}\)

\begin{Shaded}
\begin{Highlighting}[]
\NormalTok{Sin(}\DecValTok{75}\NormalTok{) }\OperatorTok{*} \DecValTok{25}
\end{Highlighting}
\end{Shaded}

$\displaystyle 24.1481456572267$

Udregningen viser at borgmuren er 24.148 meter høj.

\hypertarget{opgave-4}{%
\subsection{Opgave 4}\label{opgave-4}}

Dragejægeren Lilina er kommet op på toppen af muren og lister sig videre
til den modsatte borgmur hvor der er 30 meter lodret ned til det vand
der er på denne side af borgen. Lilina skuer ud over vandet, som
strækker sig så langt øjet rækker.

• Hvor langt væk befinder horisonten sig hvis planeten Lilina befinder
sig på har en omkreds på 40.000 km?

For at vi kan finde ud af hvor horisonten er bruger jeg formel:
\(horisont = \sqrt{højde \cdot diameter + højde}\)

\begin{Shaded}
\begin{Highlighting}[]
\NormalTok{svar }\OperatorTok{=} \DecValTok{30} \OperatorTok{*} \DecValTok{40000} \OperatorTok{/}\NormalTok{ math.pi }\OperatorTok{+} \DecValTok{30}
\NormalTok{sqrt(svar)}
\end{Highlighting}
\end{Shaded}

$\displaystyle 618.062993084482$

Udregningen her viser at Lilina kan se til horisonten som er 618.062
meter væk

\hypertarget{opgave-6}{%
\subsection{Opgave 6}\label{opgave-6}}

Lilina har kæmpet sig ned til borgens køkken hvor hun er ved at blive
overmandet af de mange vrede vagter. Hun bliver nødt til at glide ned ad
slisken med køkkenaffald som danner en vinkel på 32∘ med vandret.

• Hvor langt er Lilina gledet når hun er faldet 25 cm i højden?

For at finde ud af hvor langt Lilina er gledet ned af slisken skal vi
bruge tangens.

Formel for tangens: \(tan(V)=\dfrac{modstående}{tilstødende}\)

\(tan(32)=\dfrac{25}{længde}\) \(længde = \dfrac{25}{tan(32)}\)

\begin{Shaded}
\begin{Highlighting}[]
\NormalTok{Tan(}\DecValTok{32}\NormalTok{)}
\end{Highlighting}
\end{Shaded}

$\displaystyle 0.624869351909328$

\(længde = \dfrac{25}{0,624}\)

\begin{Shaded}
\begin{Highlighting}[]
\DecValTok{25}\OperatorTok{/}\FloatTok{0.624}
\end{Highlighting}
\end{Shaded}

\begin{verbatim}
40.06410256410256
\end{verbatim}

Så Lilina gled cirka 40cm ned af slisken

\hypertarget{opgave-7}{%
\subsection{Opgave 7}\label{opgave-7}}

I den vilkårlige trekant ABC kendes siderne 𝑎 = 4,72 og 𝑐 = 7,35 samt
vinkel 𝐵 = 56,8∘. 1. Beregn den resterende side og de resterende
vinkler.

I den vilkårlige trekant DEF kendes siden 𝑑 = 4,72 samt vinklerne 𝐸 =
56,8∘ og 𝐹 = 30∘. 2. Beregn de resterende sider og vinkel.

I den vilkårlige trekant GHI kendes siderne 𝑔 = 5 og ℎ = 4. 3. Hvis
vinklen 𝐻 = 45∘, hvad kan vinklen 𝐼 så være? 4. Er det en trekant hvis 𝐻
= 60∘? 5. Hvad er den største værdi som 𝐻 kan være? Hvad er vinkel 𝐺 i
dette tilfælde?

\begin{enumerate}
\def\labelenumi{\arabic{enumi}.}
\tightlist
\item
  Beregn den resterende side og de resterende vinkler. For at finde
  siden b bruger jeg pythagoras læresætning som fortæller os at
  \(a^2 + b^2 = c^2\)
\end{enumerate}

\begin{Shaded}
\begin{Highlighting}[]
\FloatTok{4.72}\OperatorTok{**}\DecValTok{2} \OperatorTok{+} \FloatTok{7.35}\OperatorTok{**}\DecValTok{2}
\end{Highlighting}
\end{Shaded}

\begin{verbatim}
76.30089999999998
\end{verbatim}

\begin{Shaded}
\begin{Highlighting}[]
\NormalTok{sqrt(}\FloatTok{76.30089999999998}\NormalTok{)}
\end{Highlighting}
\end{Shaded}

$\displaystyle 8.73503863757911$

Efter viste udregninger ved vi at siden b = 8.73503863757911

Så finder vi siden b vha. cosinusrelationer:
\(\sqrt{a^2 + c^2 - 2 \cdot a \cdot c \cdot cos(B)}\)

\begin{Shaded}
\begin{Highlighting}[]
\NormalTok{sqrt(}\FloatTok{4.72}\OperatorTok{**}\DecValTok{2}\OperatorTok{+}\FloatTok{7.35}\OperatorTok{**}\DecValTok{2}\OperatorTok{{-}}\DecValTok{2}\OperatorTok{*}\FloatTok{4.72}\OperatorTok{*}\FloatTok{7.35}\OperatorTok{*}\NormalTok{Cos(}\FloatTok{56.8}\NormalTok{))}
\end{Highlighting}
\end{Shaded}

$\displaystyle 6.18940815435473$

Så siden b = 6.189

For så at finde vinkel A bruger vi igen cosinusrelationer:
\(cos(A)=\dfrac{b^2 + c^2 - a^2}{2bc}\)

\begin{Shaded}
\begin{Highlighting}[]
\NormalTok{aCos((}\FloatTok{6.189408}\OperatorTok{**}\DecValTok{2}\OperatorTok{+}\FloatTok{7.35}\OperatorTok{**}\DecValTok{2}\OperatorTok{{-}}\FloatTok{4.72}\OperatorTok{**}\DecValTok{2}\NormalTok{)}\OperatorTok{/}\NormalTok{(}\DecValTok{2}\OperatorTok{*}\FloatTok{6.189408}\OperatorTok{*}\FloatTok{7.35}\NormalTok{))}
\end{Highlighting}
\end{Shaded}

$\displaystyle 39.6510830254133$

Vinkel A er så 39.651 grader

Vinkel C finder vi på baugrund af vores viden om at der er 180 grader i
en trekant.

\begin{Shaded}
\begin{Highlighting}[]
\DecValTok{180} \OperatorTok{{-}} \FloatTok{39.6510830254133} \OperatorTok{{-}} \FloatTok{56.8}
\end{Highlighting}
\end{Shaded}

\begin{verbatim}
83.5489169745867
\end{verbatim}

Derfor er vinkel C 83.5489169745867 grader

\begin{enumerate}
\def\labelenumi{\arabic{enumi}.}
\setcounter{enumi}{1}
\tightlist
\item
  Beregn de resterende sider og vinkel.
\end{enumerate}

Vi starter med at finde den sidste vinkel. D. Det gør vi igen fordi vi
ved der er 180 grader i en trekant.

\begin{Shaded}
\begin{Highlighting}[]
\DecValTok{180} \OperatorTok{{-}} \FloatTok{56.8} \OperatorTok{{-}} \DecValTok{30}
\end{Highlighting}
\end{Shaded}

\begin{verbatim}
93.2
\end{verbatim}

Nu har vi alle vinklerne i vores trekant og vi kan begynde at beregne de
sidste sider.

nu regener vi siden e

\begin{Shaded}
\begin{Highlighting}[]
\FloatTok{4.72}\OperatorTok{*}\NormalTok{Sin(}\FloatTok{93.2}\NormalTok{)}\OperatorTok{/}\NormalTok{Sin(}\DecValTok{30}\NormalTok{)}
\end{Highlighting}
\end{Shaded}

$\displaystyle 9.4252808138779$

Siden er er derfor 9.4252808138779

Så regner vi den sidste side som er siden f

\begin{Shaded}
\begin{Highlighting}[]
\FloatTok{4.72}\OperatorTok{*}\NormalTok{Sin(}\FloatTok{56.8}\NormalTok{)}\OperatorTok{/}\NormalTok{Sin(}\DecValTok{30}\NormalTok{)}
\end{Highlighting}
\end{Shaded}

$\displaystyle 7.8990551190526$

Siden f er derved 7.8990551190526

\begin{enumerate}
\def\labelenumi{\arabic{enumi}.}
\setcounter{enumi}{2}
\tightlist
\item
  Hvis vinklen 𝐻 = 45∘, hvad kan vinklen 𝐼 så være?
\end{enumerate}

Vi finder siden I med cosinusrelationerne igen:
\(cos(A)=\dfrac{b^2 + c^2 - a^2}{2bc}\)

\begin{Shaded}
\begin{Highlighting}[]
\NormalTok{aCos((}\DecValTok{4}\OperatorTok{**}\DecValTok{2}\OperatorTok{+}\DecValTok{5}\OperatorTok{**}\DecValTok{2}\OperatorTok{{-}}\FloatTok{5.406363}\OperatorTok{**}\DecValTok{2}\NormalTok{)}\OperatorTok{/}\NormalTok{(}\DecValTok{2}\OperatorTok{*}\DecValTok{4}\OperatorTok{*}\DecValTok{5}\NormalTok{))}
\end{Highlighting}
\end{Shaded}

$\displaystyle 72.8855733293959$

Vinkel også være:

\begin{Shaded}
\begin{Highlighting}[]
\NormalTok{aCos((}\DecValTok{4}\OperatorTok{**}\DecValTok{2}\OperatorTok{+}\DecValTok{5}\OperatorTok{**}\DecValTok{2}\OperatorTok{{-}}\FloatTok{1.664705}\OperatorTok{**}\DecValTok{2}\NormalTok{)}\OperatorTok{/}\NormalTok{(}\DecValTok{2}\OperatorTok{*}\DecValTok{4}\OperatorTok{*}\DecValTok{5}\NormalTok{))}
\end{Highlighting}
\end{Shaded}

$\displaystyle 17.1144297194546$

Så vinkel I kan være 72.885 grader og 17.114 grader



\end{document}
